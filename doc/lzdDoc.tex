\documentclass{article}
\usepackage{graphics,eurosym,latexsym}
\usepackage{listings}
\lstset{columns=fixed,basicstyle=\ttfamily,numbers=left,numberstyle=\tiny,stepnumber=5,breaklines=true}
\usepackage{times}
\usepackage[round]{natbib}
\bibliographystyle{plainnat}
\oddsidemargin=0cm
\evensidemargin=0cm
\newcommand{\be}{\begin{enumerate}}
\newcommand{\ee}{\end{enumerate}}
\newcommand{\bi}{\begin{itemize}}
\newcommand{\ei}{\end{itemize}}
\newcommand{\I}{\item}
\newcommand{\ty}{\texttt}
\newcommand{\kr}{K_{\rm r}}
\textwidth=16cm
\textheight=23cm
\begin{document}
\title{\ty{lzd}, v. 0.7, Lempel-Ziv Decomposition}
\author{Bernhard Haubold}
\date{February 29, 2024}
\maketitle
\section{Introduction}
The Lempel-Ziv decomposition is central to data compression algorithms
and the program \ty{lzd} serves as a didactic tool to generate the
decomposition, which divides a string into repeating
subunits~\citep{ziv77:uni}. More formally, let $S$ be a string, then
starting at the first position of $S$, $S[1]$, look for the longest
prefix of $S[1..]$ that is repeated somewhere to the left of
$S[1]$. If there is no repeat, as in the case of the first character,
the character itself is the factor. The factor is skipped, and the
search for the next factor starts at $S[2]$, and so on. For example,
let $S=\ty{CCCTCTGCGA}$, the decomposition is
\[
\ty{C.CC.T.CT.G.C.G.A}
\]
We can think of these factors as being ``left''-repeats, as the repeat
starting at $S[i]$ is always located to the left of $S[i]$, while the
remainder of the string to the right is ignored.

The program \ty{lzd} calculates the Lempel-Ziv decomposition using the
algorithm by \citet{cro08:sim}, and it can also generate and a table
like the one shown in their Figure 1.

\section{Tutorial}
The input string for \ty{lzd} needs to be in FASTA format, which is
widely used in bioinformatics and consists of a header line beginning
with \verb+>+, followed by the text on an arbitrary
number of non-empty lines. For an example, here is the sequence
factorized above:
\begin{verbatim}
$ cat data/test.fasta
>TestSeq
CCCTCTGCGA
\end{verbatim}
To factorize it, enter
\begin{verbatim}
$ lzd data/test.fasta 
C.CC.T.CT.G.C.G.A
\end{verbatim}
To factorize the example string used by \citet{cro08:sim}, enter
\begin{verbatim}
lzd data/algPaper.fasta 
a.b.b.a.abb.baa.ab.ab
\end{verbatim}
To also reproduce Figure 1 by \citet{cro08:sim}, generate tabular
output with the \ty{-t} option:
\begin{verbatim}
$ lzd -t data/algPaper.fasta 
i       w[i]    sa[i]   lcp[i]  lpf[i]
0       a       8       0       0
1       b       9       2       0
2       b       3       0       1
3       a       12      1       1
4       a       10      1       3
5       b       0       0       2
6       b       4       3       4
7       b       13      0       3
8       a       7       0       2
9       a       2       0       3
10      a       11      2       2
11      b       6       1       2
12      a       1       0       2
13      b       5       2       1
a.b.b.a.abb.baa.ab.ab
\end{verbatim}
For nicer typesetting the table can be rendered in \LaTeX{} to give
Table~\ref{tab:lzd}.
\begin{verbatim}
$ lzd -t -l data/algPaper.fasta > tab.tex
\end{verbatim}
\begin{table}
  \caption{The data tables underlying the Lempel-Ziv decomposition of
    the example string \ty{abbaabbbaaabab}; the actual decomposition
    is shown at the bottom of the table. The details of the table's
    contents are explained by \citet{cro08:sim}.}\label{tab:lzd}
  \begin{center}
    \begin{center}
\begin{tabular}{rcrrrrl}\hline
$i$ & $\mathtt{w}[i]$ & $\mathtt{sa}[i]$ & $\mathtt{lcp}[i]$ & $\mathtt{isa}[i]$ & $\mathtt{lpf}[i]$ & $\mathtt{t}[\mathtt{sa}[i]]]$\\\hline
0 & $\mathtt{a}$ & 8 & -1 & 5 & 0 & $\mathtt{aaabab}$\\
1 & $\mathtt{b}$ & 9 & 2 & 12 & 0 & $\mathtt{aabab}$\\
2 & $\mathtt{b}$ & 3 & 3 & 9 & 1 & $\mathtt{aabbbaaabab}$\\
3 & $\mathtt{a}$ & 12 & 1 & 2 & 1 & $\mathtt{ab}$\\
4 & $\mathtt{a}$ & 10 & 2 & 6 & 3 & $\mathtt{abab}$\\
5 & $\mathtt{b}$ & 0 & 2 & 13 & 2 & $\mathtt{abbaabbbaaabab}$\\
6 & $\mathtt{b}$ & 4 & 3 & 11 & 4 & $\mathtt{abbbaaabab}$\\
7 & $\mathtt{b}$ & 13 & 0 & 8 & 3 & $\mathtt{b}$\\
8 & $\mathtt{a}$ & 7 & 1 & 0 & 2 & $\mathtt{baaabab}$\\
9 & $\mathtt{a}$ & 2 & 3 & 1 & 3 & $\mathtt{baabbbaaabab}$\\
10 & $\mathtt{a}$ & 11 & 2 & 4 & 2 & $\mathtt{bab}$\\
11 & $\mathtt{b}$ & 6 & 1 & 10 & 2 & $\mathtt{bbaaabab}$\\
12 & $\mathtt{a}$ & 1 & 4 & 3 & 2 & $\mathtt{bbaabbbaaabab}$\\
13 & $\mathtt{b}$ & 5 & 2 & 7 & 1 & $\mathtt{bbbaaabab}$\\
\hline\end{tabular}
\end{center}
\[
\mathtt{a}\cdot\mathtt{b}\cdot\mathtt{b}\cdot\mathtt{a}\cdot\mathtt{abb}\cdot\mathtt{baa}\cdot\mathtt{ab}\cdot\mathtt{ab}
\]

  \end{center}
\end{table}
Finally, there is also the option to print the match factors used by
\citet{pir19:hig}.
\begin{verbatim}
$ lzd -m data/algPaper.fasta
abb.aab.bbaa.ab.ab
\end{verbatim}
\section{Listings}
The following listings document central parts of \ty{lzd}.
\lstset{language=c}
\subsection{The Driver Program: \ty{lzd.c}}
\lstinputlisting{../src/lzd.c}
\subsection{Calculating the Enhanced Suffix Array: \ty{esa.c}}
\lstinputlisting{../src/esa.c}
\subsection{Lampel-Ziv Factorization: \ty{factor.c}}
\lstinputlisting{../src/factor.c}
\bibliography{ref}
\end{document}

